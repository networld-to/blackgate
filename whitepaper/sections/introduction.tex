% -*- root: ../distributed_hosting_whitepaper.tex -*-

This part gives a motivation and a general overview about the
\textit{Distributed Hosting Engine}. It introduces some concepts that will be
explained in more detail in the next part. If you are looking for a detailed
specification you can directly go to Part \ref{part:specifications}.

The here presented \textit{Distributed Hosting Engine} is designed with the
following properties in mind:

\begin{description}
\item[Self-Managed Overlay Networks] Hosted pages are distributed between
nodes, dependent on different metrics. This increases not only performance,
but makes the network also robust against malicious nodes, failures and other
expected or unexpected changes.
\item[Hosting from Everywhere] By design pages can be hosted behind NATs (no
public IP) and Firewalls, making the participation of mobile nodes possible.
\item[Privacy Protecting] The blockchain, nor the protocol, does expose
IPs, locations or any other privacy critical data.
\item[Distributed] There is no central authority, nor should there be any
central infrastructure.
\end{description}

\section{Terminology}\index{Terminology}

\begin{description}
\item[Blockchain] A chain of blocks that origin from the same genesis block.
\item[Block] A collection of transactions that are bundled together to a
block. Each block links exactly to one previous block (except the genesis
block). All transactions that are part of a block are confirmed by the network.
\item[Genesis Block] First block in a blockchain. Does not reference a
previous block.
\item[Transaction] There are two types of transactions, UPDATE and CLONE.
UPDATE transactions can only be issued by the content owner. CLONE
transactions are issued by everybody that decides to host a page. Both
transactions include a proof of ownership of the included hostname.
\item[Protocol] A set of rules how nodes communicate with each other and how
the transmitted packages look like. In the wider sense also how multiple
blockchains are managed.
\item[Content Owner] The entity that currently owns a page with related
blockchain. Ownership is reflected by a crypographical keypair.
\item[Co-Host] An entity that hosts a page with related blockchain that it
does not own.
\item[Website] A website typically includes at least html, css, javascript
files and images.
\item[Snapshot] A version of a website that is distributed from the content
creator to all co-hosts. A new snapshot is announced in a UPDATE transaction
and co-host upgrades are announces by CLONE transactions.
\item[Node] A host that is able to communicate with other participants in the
network. Generally a node is capable to manage blockchains and distributed
snapshots.
\item[Overlay Network] An overlay network is logical network structure on top
of physical network. The main overlay networks are the connections between
different nodes that share the same blockchain. Another overlay network is the
filesharing peer-to-peer network.
\end{description}

\section{Hosting Paradigm Shift}\index{Hosting Paradigm Shift}

This Whitepaper specifies a new protocol and blockchain that will change the
hosting of semi-static pages - like blogs, company, private and news pages.
Instead of retrieving
pages from a specific location, they are hosted directly on end-user
devices. The distributed hosting algorithm and protocol, specified here, takes
care of the optimal location for each page. By taking into account different
metrics, like response times, availability and/or relevance, the perfect place
for each page will be found after some time. This not only increases the
performance for single page accesses, but reduced also network load and hence
increases the overall network health.

Additional to the increased performance, the underlying blockchain assures the
correctness and validity of each page, also, or especially, if hosted by a
random node in the network. This validation mechanism is then used to create a
reputation system, allowing blocking of malicious nodes that have a reputation
score that is lower than a threshold value.

This mechanism, together with the page distribution algorithm, makes the
network self-managing, self-healing and robust against changes.

By design the \textit{Distributed Hosting Engine} is censorship resistance,
but only under the assumption that there are enough nodes that are willing to
host the page. This moves responsibility what could be hosted from a central
authority to the collective.

From an economical point of view, the here proposed approach, allows the
creation of an interoperable hosting ecosystem. Each hosting provider acts as
clone and has no direct access to the page, but can support the user with
additional (paid or free) services and hosting space. By design a switch from
one hosting provider to another one does not affect the hosting of the page,
nor the page itself. Technical there is not even a migration happening, only a
switch from one complementary service to another one.

\section{Use Cases}\index{Use Cases}

\subsection{Simple Blog Hosting}

Jane Doe creates locally a new blog with the help of the jekyll static site
generator \cite{Preston-Werner2015}. The \textit{Distributed Hosting Engine}
software takes in the background care of the generation of a new blockchain
for this page and the distribution to a bunch of neighbour nodes. After some
time, when the page was successfully distributed, Jane switches off her
laptop. Even though offline, John can access her blog, because he has
previously cloned the page content and has the complete blockchain locally.

\subsection{Shared Access with Optional Paid Service}

John and Jane Doe share access to the same blog. Both have access to the same,
related blockchain of this page. John is not familiar with website
programming, so he decides to pay for a service in order to have a higher
level frontend that allows him to generate pages via a wizard. Jane on the
other side is a senior web developer and decides not to pay for any service,
but to develop the page by her own without using an external service and hence
for free.
