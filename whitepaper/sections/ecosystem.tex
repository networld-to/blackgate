% -*- root: ../distributed_hosting_whitepaper.tex -*-

\section{Hosting Provider}\index{Hosting Provider}

\textbf{TODO:} \textit{Explain here the role change of hosting providers. For
example: A hosting provider does provide service on top of the blockchain,
like page generators and frontends for page publications and updates. If a
hosting provider decides to host a page, it is only one clone out of many
without special role.}
\newline

By introducing the \textit{Distributed Hosting Engine} the role of hosting
providers will change. Instead of being the only entity that hosts an instance
of a page, they will be part of an overall network of hosters. The hosting
responsibility is shared between always-on devices, like servers and
partially-on devices, like laptops or desktop computers. Additional a single
hosting provider will not be able to control the distribution of the page,
because that is determined by the collective of nodes in the network. That
said, hosting providers are playing an important role in the overall
ecosystem. They not only guarantee uptime of pages by hosting them 24/7, but
can also provide additional services that abstract away from the underlying
blockchain and page development. Especially the former point is important in
the early stage of the network and/or when a new page joins the network. A
bigger network with pages that had time to distribute in an optimal way makes
dedicated hosting servers unnecessary.

\section{Complementary Services}\index{Complementary Services}

\textbf{TODO:} \textit{Figure out what alternatives services could be provided
by the ecosystem. This includes also new business ideas for new types of
startups.}
\newline

\begin{description}
\item[Hosting Space] As mentioned in the previous section, in the early stage
there is a need for 24/7 online hosting servers in order to guarantee uptime.
These \textit{Clones} are not different than any other clone, except that they
are always online.
\item[Page Generation] The countless simple page generation services that
exists today can be used to support non-technical users to host their pages.
\item[Host Wallets] Like Bitcoin Wallets this \textit{Host Wallets} manage all
pages related to one users. Not really recommended for security reasons,
because private keys are controlled by the \textit{Host Wallet} provider.
\end{description}
